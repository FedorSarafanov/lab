\documentclass[a4paper,12pt]{article}
 
\usepackage{extsizes}
\usepackage{cmap}
\usepackage[T2A]{fontenc}
\usepackage[utf8x]{inputenc}
% \usepackage[russian]{babel}
\usepackage[english, russian]{babel}
% \usepackage{newtx}
% \usepackage{cyrtimes}
\usepackage{misccorr}
\usepackage{array}
\usepackage{tabu}
\usepackage{hhline}
%%%%%%%%%%%%%%%%%%%%%%%%%%%%%%%%%%%%%%%%%%%%%%%%%%%%%%%%%%%%%%%%%%%%%%%%%%%%%%%%%%  
\usepackage{graphicx} % для вставки картинок
\graphicspath{{img/}}
\usepackage{amssymb,amsfonts,amsmath,amsthm} % математические дополнения от АМС

% \usepackage{fontspec}
% \usepackage{unicode-math}

\usepackage{indentfirst} % отделять первую строку раздела абзацным отступом тоже
\usepackage[usenames,dvipsnames]{color} % названия цветов
\usepackage{makecell}
\usepackage{multirow} % улучшенное форматирование таблиц
\usepackage{ulem} % подчеркивания
\linespread{1.3} % полуторный интервал
% \renewcommand{\rmdefault}{ftm} % Times New Roman (не работает)
\frenchspacing
\usepackage{geometry}
\geometry{left=3cm,right=2cm,top=3cm,bottom=3cm,bindingoffset=0cm}
\usepackage{titlesec}

% \definecolor{black}{rgb}{0,0,0}
% \usepackage[colorlinks, unicode, pagecolor=black]{hyperref}
% \usepackage[unicode]{hyperref} %ссылки
% \usepackage{fancyhdr} %загрузим пакет
% \pagestyle{fancy} %применим колонтитул
% \fancyhead{} %очистим хидер на всякий случай
% \fancyhead[LE,RO]{Сарафанов Ф.Г.} %номер страницы слева сверху на четных и справа на нечетных
% \fancyhead[CO, CE]{Отчёт по лабораторной работе №16}
% \fancyhead[LO,RE]{Определение ${g}$} 
% \fancyfoot{} %футер будет пустой
% \fancyfoot[CO,CE]{\thepage}
\renewcommand{\labelenumii}{\theenumii)}
\newcommand{\ddt}{$\ \pm\ 0.2\ \text{с}$}
\newcommand{\ddtv}{$\ \pm\ 0.8\ \text{с}$}
\newcommand{\ddh}{$\ \pm\ 0.1\ \text{см}$}

\usepackage{amsthm}
\newtheorem{define}{Определение}
\newtheorem{theorem}{Теорема}
\newtheorem{problem}{Задача}

\begin{document}
\pagestyle{empty}
\begin{center}
	Протокол\\
	Лабораторная работа №22\\
	\textbf{\textsc{Определение коэффициента внутреннего трения жидкости}}
\end{center}
\underline{\textbf{Приборы и оборудование}}: цилиндрический сосуд со шкалой, глицерин, секундомер, видеокамера, микрометр, шарики стальные и пластмассовые, пинцет, термометр. 

\vspace{1em}
Радиус цилиндра $R=4.63$ см, частота съемки $\omega=30$ Гц, $\Delta{t}=0.06$ с, $\Delta{R}=0.05$ мм, $\Delta{d}=0.01$ мм, $\rho_\text{пластм}=(2.00\pm0.03)$ $\frac{\text{г}}{\text{см}^3}$

\vspace{1cm}
1. Измерение микрометром диаметра шариков.
\\
%
\\
\begin{tabu} to \textwidth {|X[1c]|X[1.5c]|*{5}{X[c]|}} 
\hline
№ & Материал & \multicolumn{3}{c|}{\multirow{2}{*}{\vspace{1.5em}Диаметр шарика $d$, мм }} & $d_\text{сред}$, мм & $r_\text{сред}$, мм \\ \hhline{|~|~|-|-|-|~|~|}
 & &$d_1$, мм& $d_2$, мм&$d_3$, мм& & \\
\hline
1	&	сталь  		&	&	&	&	&	\\ 
\hline
2	&	пластмасса	&	&	&	&	&	\\ 
\hline
\end{tabu}

\vspace{1cm}
2. Определение длины пути, на котором происходит установление скорости шариков.
\\
%
\\
\begin{tabu} to \textwidth {|*{6}{X[c]|}} 
\hline
№ шарика & Материал & $L$, см & 5-20 & 20-34 & 35-50 \\ 
\hline
1	&	сталь  		&	\multirow{2}{*}{$t$, c}	&	&	&	\\ 
\hhline{--|~|---}
2	&	пластмасса	&&&&		\\ 
\hline
\end{tabu}

\vspace{1cm}
3. Определение вязкости глицерина $\displaystyle\eta=\frac{2}{9}r^2g\frac{\rho_\text{ш}-\rho_\text{ср}}{v}\cdot\frac{1}{1+2.4\frac{r}{R}}$
\\
%
\\
\begin{tabu} to \textwidth {|X[.4c]|X[1.7c]|X[1.3c]|X[0.9c]|*{3}{X[0.7c]|}X[1.2c]|X[1.2c]|X[c]|} 
\hline
\multirow{2}{*}{№} &%
							\multirow{2}{*}{Материал} &%
							\multirow{2}{*}{$r_\text{сред}$, см}&%
							\multirow{2}{*}{$L$, см }&%
							\multicolumn{3}{c|}{$t$, c} &%
							\multirow{2}{*}{$t_\text{сред}$, с }&%
							\multirow{2}{*}{$V$, см/c }&%
							\multirow{2}{*}{$\eta$, П}\\ 
\hhline{|~|~|~|~|---|~|~|~|}
 & & & & {} & {} & {} & & & \\ \hline
 1& сталь & & & {} & {} & {} & & & \\ \hline
 2& пластмасса & & & {} & {} & {} & & & \\
\hline
% 1	&	сталь  		& &	&	\multirow{2}{*}{t, c}	&	&	&	\\ 
% \hhline{--~---}
% 2	&	пластмасса	&&&&		\\ 
% \hline
\end{tabu}






\end{document}
