\documentclass[a4paper,12pt]{article}

\usepackage{cmap}
\usepackage[T2A]{fontenc}
\usepackage[utf8x]{inputenc}
\usepackage[english, russian]{babel}

\usepackage{misccorr} % в заголовках появляется точка, но при ссылке на них ее нет
\usepackage{amssymb,amsfonts,amsmath,amsthm}  
\usepackage{indentfirst}
\usepackage[usenames,dvipsnames]{color} 
\usepackage[unicode, colorlinks, urlcolor=magenta, linkcolor=black, pagecolor=black]{hyperref}
\usepackage{makecell,multirow} 
\usepackage{ulem}
\usepackage{graphicx}
\graphicspath{{img/}}
\usepackage{geometry}
\geometry{left=3cm,right=2cm,top=3cm,bottom=3cm,bindingoffset=0cm}
\usepackage{fancyhdr} 
\linespread{1.3} 
\frenchspacing 
\renewcommand{\labelenumii}{\theenumii)} 

%%%%%%%%%%%%%%%%%%%%%%%%%%%%%%%%%%%%%%%%%%%%%%%%%%%%%%%%%%%%%%%%%%%%%%%%%%%%%%%
%%%%%%%%%%%%%%%%%%%%%%%%%%%%%%%%%%%%%%%%%%%%%%%%%%%%%%%%%%%%%%%%%%%%%%%%%%%%%%%

\def\labauthor{Сарафанов Ф.Г.}
\def\labauthors{Сарафанов Ф.Г.}
\def\labnumber{22}
\def\labtheme{Определение коэффициента внутреннего трения (вязкости) жидкости}

%%%%%%%%%%%%%%%%%%%%%%%%%%%%%%%%%%%%%%%%%%%%%%%%%%%%%%%%%%%%%%%%%%%%%%%%%%%%%%%
\input{addons/lab-kol} % колинтулы на страницах
%%%%%%%%%%%%%%%%%%%%%%%%%%%%%%%%%%%%%%%%%%%%%%%%%%%%%%%%%%%%%%%%%%%%%%%%%%%%%%%
\usepackage{float}
\usepackage[mode=buildnew]{standalone}
\usepackage{tikz} 

\usepackage{tikz,csvsimple}
\usetikzlibrary{scopes}
\usetikzlibrary{%
     decorations.pathreplacing,%
     decorations.pathmorphing,%
    patterns,%
    calc,%
    scopes,%
    arrows,%
    % arrows.spaced,%
}
\makeatletter
\newif\if@gather@prefix 
\preto\place@tag@gather{% 
  \if@gather@prefix\iftagsleft@ 
    \kern-\gdisplaywidth@ 
    \rlap{\gather@prefix}% 
    \kern\gdisplaywidth@ 
  \fi\fi 
} 
\appto\place@tag@gather{% 
  \if@gather@prefix\iftagsleft@\else 
    \kern-\displaywidth 
    \rlap{\gather@prefix}% 
    \kern\displaywidth 
  \fi\fi 
  \global\@gather@prefixfalse 
} 
\preto\place@tag{% 
  \if@gather@prefix\iftagsleft@ 
    \kern-\gdisplaywidth@ 
    \rlap{\gather@prefix}% 
    \kern\displaywidth@ 
  \fi\fi 
} 
\appto\place@tag{% 
  \if@gather@prefix\iftagsleft@\else 
    \kern-\displaywidth 
    \rlap{\gather@prefix}% 
    \kern\displaywidth 
  \fi\fi 
  \global\@gather@prefixfalse 
} 
\newcommand*{\beforetext}[1]{% 
  \ifmeasuring@\else
  \gdef\gather@prefix{#1}% 
  \global\@gather@prefixtrue 
  \fi
} 
\makeatother

\begin{document}

\input{addons/lab-titlepage}

% \tableofcontents
% \newpage
% \section{Снятая зависимость}
\def\rhos{\rho_\text{шар}}
\def\rhoz{\rho_\text{жид}}
\def\Q{g\frac{\rhos-\rhoz}{\rhos}}
	\def\K{\frac{2}{9}gr^2\frac{\rhos-\rhoz}{\eta}}
\def\C{\Q-kv}%
\begin{gather}
	\beforetext{Запишем II закон Ньютона} m\vec{a}=\vec{F}_\text{тр}+m\vec{g}+\vec{F}_\text{арх}\\
	\beforetext{Где сила Стокса} F_\text{тр}=6\pi\eta{r}v\\
	\label{eq:ox}\beforetext{В проекции на $x$:}ma=mg-6\pi\eta{r}v-\rhoz{}gV\\
	\beforetext{}V=\frac{4}{3}\pi{r^3}, m=\rhos{}V\\
	\beforetext{Перепишем (\ref{eq:ox}):}ma=mg-6\pi\eta{r}v-\rhoz{}gV%
	%
	\\
	\beforetext{Введем константу $k$:}k=6\pi\eta{r}\cdot \frac{1}{V\rhos}=\frac{6\pi\eta{r}}{m}\\
	\rhos{V}\frac{dv}{dt}=gV(\rhos-\rhoz)-kv \cdot V{\rhos}\\
	\frac{dv}{\C}=dt\\
	\beforetext{Замена переменной:   } \ \ c=\C\\
	dc=-k\,dv\\
	-\frac{1}{k}\int^{\Q-kv(t)}_{\Q-kv_0} \frac{dc}{c} = \int_{0}^{t} dt\\
	\ln\left(\frac{\Q-kv(t)}{\Q-kv_0}\right)=-kt\\
	\frac{\Q-kv(t)}{\Q-kv_0}=e^{-kt}\\
	{\Q-kv(t)}=e^{-kt}({\Q-kv_0})\\
	kv(t)=e^{-kt}kv_0-e^{-kt}\Q+\Q\\
	kv(t)=\Q(1-e^{-kt})+e^{-kt}kv_0\\
	v(t)=\Q\frac{1-e^{-kt}}{k}+e^{-kt}v_0\\
	v(t)=Vg\frac{\rhos-\rhoz}{6\pi\eta{r}}({1-e^{-kt}})+e^{-kt}v_0\\
	v(t)=\frac{2}{9}gr^2\frac{\rhos-\rhoz}{\eta}\cdot({1-e^{-kt}})+e^{-kt}v_0, \text{ где } k=\frac{6\pi\eta{r}}{m}.\\%
	%
	%
	s(t)=\int_0^{s(t)}v(t)\ dt = \K\cdot (\frac{e^{-kt}}{k}+t)+\frac{e^{-kt}v_0}{k}
%
\end{gather}

Для движения без начальной скорости
\begin{gather}
	v(t)=\frac{2}{9}gr^2\frac{\rhos-\rhoz}{\eta}\cdot({1-e^{-kt}})\\
	s(t)=\K\cdot (\frac{e^{-kt}}{k}+t)\\
	1-e^{-kt^*}=\alpha=0.95\\
	e^{-kt^*}=\beta=0.05\\
	-kt^*=\ln{\beta}\\
	t^*=-\frac{1}{k}\ln{\beta}=-\frac{m}{6\pi\eta{r}}\ln{\beta}\\
	s^*=\K\cdot(\frac{\beta-\ln{\beta}}{k})=%
	\frac{m}{27\pi}gr\frac{\rhos-\rhoz}{\eta^2}(\beta-\ln{\beta})
\end{gather}

$t^*$ -- время установления, когда скорость будет отличаться от установившейся $v_\text{уст}$ на бесконечности не более чем в $\alpha$ раз.

$s^*$ -- соответственно путь установления, когда скорость отличается от установившейся $v_\text{уст}$ на бесконечности не более чем в $\alpha$ раз.

\begin{gather}
	v_\text{уст}=\lim_{t \to \infty}{\left[\K\cdot(1-e^{-kt})\right]}=\K
\end{gather}
\end{document}
