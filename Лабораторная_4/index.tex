%%%%%%%%%%%%%%%%%%%%%%%%%%%%%%%%%%%%%%%%%%%%%%%%%%%%%%%%%%%%%%%%%%%%%%%%%%%%%%%
\input{addons/lab-head}  % преамбула
%%%%%%%%%%%%%%%%%%%%%%%%%%%%%%%%%%%%%%%%%%%%%%%%%%%%%%%%%%%%%%%%%%%%%%%%%%%%%%%

\newcommand{\labauthor}{Сарафанов Ф.\,Г.}
\newcommand{\labauthors}{Сарафанов Ф.\,Г.}
% \newcommand{\labauthors}{Сарафанов Ф.\,Г., Сидоров Д.\,А.}
\newcommand{\labnumber}{17}
\newcommand{\labtheme}{Осциллограф}


\newcommand{\ddt}{$\ \pm\ 0.2\ \text{с}$}
\newcommand{\ddtv}{$\ \pm\ 0.8\ \text{с}$}
\newcommand{\ddh}{$\ \pm\ 0.1\ \text{см}$}
\newcommand{\dm}{\Delta{}m}
\newcommand{\Dh}{\Delta{}x}
\newcommand{\Dl}{\Delta{}(\lambda)}
\newcommand{\dmsr}{<\Delta{}m>}
\newcommand{\el}{\varepsilon(\lambda)}

\usetikzlibrary{%
    decorations.pathreplacing,%
    decorations.pathmorphing,%
    arrows,%
    patterns
}
\newcommand{\Scale}{1}
\newcommand{\lft}{9}
\newcommand{\rft}{10.43*1.5}
\newcommand{\Xstep}{1.5}
\newcommand{\Ystep}{1.5*20}
\newcommand{\Radius}{0.1}
\newcommand{\Color}{black}


%%%%%%%%%%%%%%%%%%%%%%%%%%%%%%%%%%%%%%%%%%%%%%%%%%%%%%%%%%%%%%%%%%%%%%%%%%%%%%%
\input{addons/lab-kol} % колинтулы на страницах
%%%%%%%%%%%%%%%%%%%%%%%%%%%%%%%%%%%%%%%%%%%%%%%%%%%%%%%%%%%%%%%%%%%%%%%%%%%%%%%

\begin{document}

\input{addons/lab-titlepage}

\tableofcontents

\newpage
\section{Описание лабораторной установки}

\textbf{Цель работы:} изучение характера движения заряженных частиц в однородном магнитном поле и определение удельного заряда электрона методом магнитной фокусировки и методом отклонения в известных полях.

\textbf{Оборудование:}
экспериментальная установка (ЭЛТ и блок питания), коммутатор, амперметр постоянного тока, источник питания постоянного тока 

\textbf{Приборные погрешности:} $\Delta{U}=62.6\ \text{В}$, $\Delta{I}=0.015\ \text{А}$, $\Delta{h}=0.01\ \text{м}$. 

\begin{figure}[ht!]
	\centering
	\includegraphics[width=0.8\textwidth]{lissajous3.png}
	\caption{Фигуры Лиссажу для $\frac{m}{n}=1;2;3;4;\frac{5}{2};\frac{5}{3}$}
	\label{fig:cxem}
\end{figure}

На рисунке (\ref{fig:cxem}) изображена лабораторная установка, начиная с второго анода электронной пушки. Напряжение второго анода $U_a$ регулируется в пределах $700\div1700$ вольт, изменяя продольную скорость электронов на вылете из пушки.

На отклоняющие пластины подается (не одновременно во время опытов) переменное напряжение $U_\text{в}$ ($U_\text{г}$) с эффективным значением $75$ вольт и частотой $50$ Гц. 

Вокруг трубки намотан соленоид с диаметром 7 см, плотность намотки 



\section{Измерение удельного заряда электрона методом отклонения земным магнитным полем}

В лабораторной работе исследуется .

Погрешности, используемые в работе: 

Запишем :
\begin{EqSystem}
\end{EqSystem}

Спроецируем на ось X, направленную :
\begin{EqSystem}
	\label{eq:}
\end{EqSystem}

\subsection{Вывод}

В результате проделанной работы были выполнены следующие пункты.

Опровергнута гипотеза 

Снята линейная зависимость  откуда сделан вывод о .

Снята зависимость ,
для которой расчитана соответствующая погрешность (\ref{})

Оценены коэффициенты $\lambda$  и $F_0$ методом .

Изучено уравнение динамики вращательного движения (ОУДВД) и физический смысл момента инерции, а также методы его вычисления.

Рассчитано значение коэффициента 

Определена правильность определения

Сравнение , полученного разными способами, показывает: в пределах погрешностей измерений можно утверждать следующее: 


В пределах погрешностей измерений были построены графики зависимостей.

В работе рассчитаны погрешности для всех косвенных измерений, размеры прямоугольников ошибок. 

Все точки на графиках укладываются на  теоретические графики в пределах размеров их прямоугольников ошибок.

Подтверждена 

\begin{figure}[h!]
	\centering
	% \includegraphics[]{}	
	\begin{tikzpicture}
	\hspace{-1cm}
		\input{img/regions}		
	\end{tikzpicture}
	\caption{Caption here}
	\label{fig:figure1}
\end{figure}

% \newpage
% \section*{Приложение 1. Графики зависимостей} % (fold)
% \label{sec:figures}

% section figures (end)

\end{document}
