\documentclass[a4paper,12pt]{article}
 
\usepackage{extsizes}
\usepackage{cmap}
\usepackage[T2A]{fontenc}
\usepackage[utf8x]{inputenc}
% \usepackage[russian]{babel}
\usepackage[english, russian]{babel}
% \usepackage{newtx}
% \usepackage{cyrtimes}
\usepackage{misccorr}

%%%%%%%%%%%%%%%%%%%%%%%%%%%%%%%%%%%%%%%%%%%%%%%%%%%%%%%%%%%%%%%%%%%%%%%%%%%%%%%%%%  
\usepackage{graphicx} % для вставки картинок
\graphicspath{{img/}}
\usepackage{amssymb,amsfonts,amsmath,amsthm} % математические дополнения от АМС

% \usepackage{fontspec}
% \usepackage{unicode-math}

\usepackage{indentfirst} % отделять первую строку раздела абзацным отступом тоже
\usepackage[usenames,dvipsnames]{color} % названия цветов
\usepackage{makecell}
\usepackage{multirow} % улучшенное форматирование таблиц
\usepackage{ulem} % подчеркивания
\linespread{1.3} % полуторный интервал
% \renewcommand{\rmdefault}{ftm} % Times New Roman (не работает)
\frenchspacing
\usepackage{geometry}
\geometry{left=3cm,right=2cm,top=3cm,bottom=3cm,bindingoffset=0cm}
\usepackage{titlesec}

% \definecolor{black}{rgb}{0,0,0}
% \usepackage[colorlinks, unicode, pagecolor=black]{hyperref}
% \usepackage[unicode]{hyperref} %ссылки
\usepackage{fancyhdr} %загрузим пакет
\pagestyle{fancy} %применим колонтитул
\fancyhead{} %очистим хидер на всякий случай
\fancyhead[LE,RO]{Сарафанов Ф.Г.} %номер страницы слева сверху на четных и справа на нечетных
\fancyhead[CO, CE]{Отчёт по лабораторной работе №16}
\fancyhead[LO,RE]{Определение ${g}$} 
\fancyfoot{} %футер будет пустой
\fancyfoot[CO,CE]{\thepage}
\renewcommand{\labelenumii}{\theenumii)}
\newcommand{\ddt}{$\ \pm\ 0.2\ \text{с}$}
\newcommand{\ddtv}{$\ \pm\ 0.8\ \text{с}$}
\newcommand{\ddh}{$\ \pm\ 0.1\ \text{см}$}

\usepackage{amsthm}
\newtheorem{define}{Определение}
\newtheorem{theorem}{Теорема}
\newtheorem{problem}{Задача}

\begin{document}

% \section{}

\section{Отчёт по лабораторной работе №12}

\subsection{Физические основы лабораторной работы}

Для получения ускорения свободного падения при помощи математического маятника используется установка из маятника с длиной нити, много большей радиуса груза (шарика), зеркальной шкалы,  градусной шкалы отклонения груза $\phi$ и секундомера.

Нить предполагается невесомой и нерастяжимой, пренебрегается силами трения. Тогда можно рассмотреть движение маятника в поле консервативной силы - силы гравитации и найти ${g}$.

Маятник двигается по закону 
\begin{equation}
	\label{eq-phi}
	\frac{d^2\phi}{dt^2}+\frac{g}{l}\cdot{}sin\,\phi=0
\end{equation}

Для \textit{малых углов} можно считать $sin\,\phi=\phi$. Тогда уравнение (\ref{eq-phi}) будет уравнением гармонического осциллятора, решение которого имеет вид 

\begin{equation}
	\label{eq-garm}
	\phi=\phi_0\,sin\,(\omega{}t+\alpha),
\end{equation}

Где $\phi_0$ -- амплитуда колебаний, $\alpha$ -- начальная фаза, $\omega=\sqrt{\frac{g}{l}}$ -- частота колебаний. Из соотношения $T=\frac{2\pi}{\omega}$ следует

\begin{equation}
	\label{g_lt}
	T=2\pi\sqrt{\frac{l}{g}}
\end{equation}

Из (\ref{g_lt}) можно выразить ускорение свободного падения (\ref{gg}):

\begin{equation}
\label{gg}
	g=4\pi^2\frac{l}{T^2}
\end{equation}

Но так как $l$ напрямую замерить сложно (от точки подвеса до центра тяжести груза), необходимо проводить несколько опытов с разными высотами, чтобы избавиться от необходимости точного измерения длины нити и перейти к измерению \textbf{разности} длин, которую можно замерить гораздо более точно(\ref{gtt}). При измерении разности можно брать любые (одинаковые относительно шарика) длины нити. Удобно взять наинизшую точку груза (шарика). Обозначим такие длины $h_2$ и $h_1$.

\begin{equation}
\label{gtt}
	T_1^2=4\pi^2\frac{l_1}{g},\ \ T_2^2=4\pi^2\frac{l_2}{g},\ \ \Rightarrow g=4\pi^2\frac{h_2-h_1}{T_2^2-T_1^2},\text{ где } T=\frac{t}{n}
\end{equation}

\subsection{Величина малых углов}

В описании физического смысла лабораторной работы была сделана оговорка о малых углах $\phi$.
Стоит отметить, что градусная шкала проградуирована в градусах, с ценой деления 1 градус. Таким образом, погрешность задания угла отклонения $\Delta\,\phi$ можно считать равной половине цены деления -- $\Delta\,\phi\approx0.0087\,\text{рад}$.

Тогда необходимо решить уравнение

\begin{equation}
\label{sin}
	sin\,\phi_0=\phi_0-\Delta\,\phi
\end{equation}

Решением (\ref{sin}) является $\phi_0\approx0.34\,\text{рад}\approx20^{\circ}$. Для $\phi\le\phi_0$ будет выполняться формула гармонического осциллятора (\ref{eq-garm}), а значит и формулы периода колебаний математического маятника (\ref{g_lt}), и все выведенные из (\ref{g_lt}) формулы (\ref{gg}, \ref{gtt}, \ldots)

Такое утверждение можно проверить экспериментально. Найдем зависимость $T(\phi)$ для разных углов отклонения, при одинаковом количестве колебаний и высоте нити маятника.

\begin{table}[h]
	\begin{center}
	\begin{tabular}{|c|c|c|c|c|c|c|}
\hline
$n$ & $\phi$, ${\,}^{\circ}$ & $5^{\circ}\pm0.5^{\circ}$ & $10^{\circ}\pm0.5^{\circ}$ & $15^{\circ}\pm0.5^{\circ}$ & $20^{\circ}\pm0.5^{\circ}$ & $40^{\circ}\pm0.5^{\circ}$ \\
\hline
20 & \text{$t$, c}&43$\pm$0.2 \text{с} & 43.4$\pm$0.2 \text{с}&44.6$\pm$0.2 \text{с}&43.8$\pm$0.2 \text{с}&48$\pm$0.2 \text{с}\\
 & $T_1=\frac{t}{n}$, с & 8.6$\pm$0.2 \text{с} & 4.3$\pm$0.2 \text{с} &2.97$\pm$0.2 \text{с} &2.19$\pm$0.2 \text{с} &2.4$\pm$0.2 \text{с} \\
\hline
		\end{tabular}
	\end{center}
\caption{\label{tab:t-phi}Зависимость периода математического маятника от угла отклонения, $T(\phi)$}
\end{table} 

Полученные результаты нанесем на график с учетом погрешностей измерения времени и углов (см. рис.\ \ref{fig1}). Легко заметить,что при $\phi>5^{\circ}$ значение периода сильно уменьшается  --- это означает, что применение формулы математического маятника (и всех из неё выведенных) на данных углах невозможно, так как предполагает период константой. График сужает утверждение $T=const$ на углы $\phi<\phi_0, \phi_0=5^{\circ}$
\newpage

\begin{figure}[h]
\begin{center}
\includegraphics*[width=1\textwidth]{img/phi_2.png}
\caption{\label{fig1}График зависимости $T(\phi)$}
\end{center}
\end{figure}



\subsection{Определение ускорения свободного падения ${g}$}
% Определение минимального количества колебаний маятника, при котором относительная погрешность вычисления ${g}$ меньше 1\%
Непосредственно замерить однократный период колебаний маятника точно достаточно сложно. Можно замерить $n$ полных колебаний маятника и их время: тогда можно ввести формулу

\begin{equation}
\label{Ttn}
	T=\frac{t}{n}
\end{equation}

Рассчитаем, для какого количества $n$ колебаний маятника $\varepsilon\,(g)\le1\%$. Для этого выведем формулу относительной погрешности $\varepsilon\,(g)$, где позже выразим $n$ из формулы (\ref{Ttn}):

\begin{eqnarray}
\label{n-rash}
\varepsilon\,(g)=\varepsilon({4\pi^2\frac{h_2-h_1}{t_2^2-t_1^2}})=\varepsilon({\frac{h_2-h_1}{t_2^2-t_1^2}})=\frac{\Delta(h_2-h_1)}{h_2-h_1}+\frac{\Delta(t_2^2-t_1^2)}{t_2^2-t_1^2}=\\\nonumber=\frac{2\Delta{h}}{h_2-h_1}+\frac{2\Delta{t}(t_2+t_1)}{(t_2-t_1)(t_2+t_1)}=\frac{2\Delta{h}}{h_2-h_1}+\frac{2\Delta{t}}{t_2-t_1}
\end{eqnarray}


\begin{eqnarray}
\label{n-rash2}
\varepsilon=\frac{2\Delta{h}}{h_2-h_1}+\frac{2\Delta{t}}{n\cdot(T_2-T_1)}\Longrightarrow
\frac{(\varepsilon-\frac{2\Delta{h}}{h_2-h_1})\cdot(T_2-T_1)}{2\Delta{t}}=\frac{1}{n}
\end{eqnarray}

\begin{eqnarray}
\label{n-rash3}
n_\text{расчетное}=\frac{2\Delta{t}}{(T_2-T_1)(0.01-\frac{2\Delta{H}}{h_2-h_1})}
\end{eqnarray}

Подставим в формулу (\ref{n-rash3}) экспериментальные значения $T_2$, $T_1$ при 3 экспериментах с разными высотами
$h_2$ и $h_1$.

\begin{table}[h]
\begin{center}
\begin{tabular}{|c|c|c|c|c|c|c|c|}
\hline 
\text{№} & $n$ & $h_1$, \text{см} & $t_1$, \text{с} & $T_1$, \text{с} & $h_2$, \text{см} & $t_2$, \text{с} & $T_2$, \text{с} \\
\hline
$1$ & $20$ & $3\pm0.1\text{см}$ & $6.9\pm0.2\text{с}$ & $0.35\pm0.2\text{с}$ & $95\pm0.1\text{см}$ & $39.2\pm0.2\text{с}$ & $1.9\pm0.2\text{с}$ \\
\hline
$2$ & $20$ & $3\pm0.1\text{см}$ & $8.7\pm0.2\text{с}$ & $0.44\pm0.2\text{с}$ & $115\pm0.1\text{см}$ & $42.3\pm0.2\text{с}$ & $2.2\pm0.2\text{с}$ \\
\hline
$3$ & $20$ & $3\pm0.1\text{см}$ & $10.5\pm0.2\text{с}$ & $0.52\pm0.2\text{с}$ & $125\pm0.1\text{см}$ & $45.0\pm0.2\text{с}$ & $2.3\pm0.2\text{с}$ \\
\hline
% \multicolumn{2}{|c|}{Результаты измерений} \\
\end{tabular}
\end{center}
% \caption{\label{tab:t-pogr}}%Измерения для подстановки в формулу погрешностей и вычисления $n_\text{расчетное}=\frac{2\Delta{t}}{(T_2-T_1)(0.01-\frac{2\Delta{H}}{h_2-h_1})}$ (\ref{n-rash})}
\end{table} 



\begin{center}
\begin{tabular}{|c|c|}
\hline 
\text{№} & Значение $n_\text{расчетное}$ \\
\hline
$1$ & $n_\text{расчетное}=\frac{2\cdot{0.2}}{(1.9-0.35)(0.01-\frac{2\cdot{0.1}}{95-3})}\approx{}31$ \\ \hline
$2$ & $n_\text{расчетное}=\frac{2\cdot{0.2}}{(2.2-0.44)(0.01-\frac{2\cdot{0.1}}{115-3})}\approx{}28$ \\ \hline
$3$ & $n_\text{расчетное}=\frac{2\cdot{0.2}}{(2.3-0.52)(0.01-\frac{2\cdot{0.1}}{125-3})}\approx{}27$ \\ \hline
% \multicolumn{2}{|c|}{Результаты измерений} \\
\end{tabular}
\end{center}

При вычисленных значениях необходимого числа $n$ замеряемых колебаний математического маятника относительная погрешность ${g}$, вычисленного по формуле (\ref{g}), не будет превышать 1\%.

\begin{equation}
\label{g}
	g=4\pi^2\frac{h_2-h_1}{T_2^2-T_1^2}
\end{equation}

\newpage

Проведем ряд опытов с учетом вышеуказанных измерений и вычислений, рассчитав погрешность измерения ${g}$ по формуле (\ref{n-rash}).

\begin{table}[h]
\begin{center}
\begin{tabular}{|c|c|c|c|}
\hline 
\text{№ опыта} & 1 & 2 & 3\\
\hline
n & 20 & 25 & 30 \\
\hline
$h_1$, см & 3\ddh&3\ddh&3\ddh \\
\hline
$t_1$, с & 6.94\ddt & 8.65\ddt & 10.45\ddt \\
\hline
$T_1=\frac{t_1}{n}$, с & 0.35\ddt & 0.35\ddt & 0.35\ddt \\
\hline
$h_2$, см & 95\ddh&115\ddh&125\ddh \\
\hline
$t_2$, с & 38.97\ddt & 53.86\ddt & 67.11\ddt \\
\hline
$T_2=\frac{t_1}{n}$, с & 1.95\ddt & 2.15\ddt & 2.24\ddt \\
\hline
$h_2-h_1$, см & 92\ddh&112\ddh&122\ddh \\
\hline
$T_2^2-T_1^2$, $c^2$ & 3.7\ddtv & 4.5\ddtv & 4.9\ddtv \\
\hline
$g$, $\text{см}/c^2$ & $987.74\pm1.4\%$  & $977.66\pm1.2\%$ & $986.41\pm1\%$ \\
\hline

\end{tabular}
\end{center}

\end{table} 

Рассчитаем среднее значение $g_\text{ср}$, также пересчитав его погрешность.

\begin{eqnarray*}
	\delta{g_\text{ср}}=\frac{\Delta{g_1}+\Delta{g_2}+\Delta{g_3}}{g_1+g_2+g_3}=
\frac{\varepsilon{g_1}\cdot{g_1}+\varepsilon{g_2}\cdot{g_2}+\varepsilon{g_3}\cdot{g_3}}{g_1+g_2+g_3}=\\
\\
=\frac{0.014\cdot987.74+0.012\cdot977.66+0.01\cdot986.41}{987.74+977.66+986.41}=0.012000=1.2\%	
\end{eqnarray*}

\begin{equation}
g_\text{ср} = \frac{987.74+977.66+986.41}{3}\ \pm\ 1.2\%=983.941\ \pm\ 1.2\%
\end{equation}

\subsection{Вывод}

Проведенная лабораторная работа позволила определить ускорение свободного падения - 
$983.941\ \text{cм/с}^2$ с относительной погрешностью $\delta{g}=1.2\%$  при измерении с помощью математического маятника. 

Результат $983.941\ \text{cм/с}^2$ отклоняется от табличного значения  $980\ \text{cм/с}^2$ на 0.33\% --- измерен с высокой точностью. 

Также в работе была определена величина малых углов, при которых в эксперименте выполняется формула Гюйгенса -- математического маятника.

В работе рассчитаны погрешности для всех косвенных измерений.

\subsection{Ответы на вопросы}

\subsubsection{Вопрос 1}

\textbf{При определении периода пускать в ход и останавливать секундомер можно: а) когда маятник имеет наибольшее отклонение; б) когда он проходит положение равновесия. В каком случае измерение точнее?}
\\

		Измерение будет более точным в случае  пуска и остановки секундомера при наибольшем отклонении. Объяснить это можно следующим образом.

		Запишем уравнение гармонического осциллятора (\ref{eq-garm}, \ref{eq-garm2})

		\begin{equation}
			\label{eq-garm2}
			\phi=\phi_0\,sin\,(\omega{}t+\alpha)
		\end{equation}

		И продифференцируем его (\ref{eq:w-garm}). Производная $\dot{\phi}$ есть угловая скорость по определению.

		\begin{equation}
			\label{eq:w-garm}
			\dot{\phi}=\phi_0\,\omega\,cos\,(\omega{}t+\alpha)
		\end{equation}

		Из начальных условий ($\phi(t=0)=0$) найдем $\alpha=\frac{\pi}{2}$. Отметим, что $|\phi_0|\ll\pi$. 

		Тогда ясно, что максимальная скорость будет при максимальном значении $cos\,(\omega{}t+\alpha)=1\ \Longrightarrow\ \omega{}t+\alpha=\pi$, то есть при $t_{v_{max}}=\frac{\pi}{2\omega}$. Отметим, что подставив в (\ref{eq-garm2}) $t_{v_{max}}$, мы получим значение $\phi$ в этот момент, равное 0. 

		Следовательно, в положении равновесия угловая (а значит, и линейная) скорость максимальна, а в положениях максимального отклонения она равна 0.

		Тогда можно сказать, что в районе максимального отклонения за время $\Delta\,t$ срабатывания секундомера маятник пройдет гораздо меньший путь, чем в положении равновесия, а значит и ошибка измерения времени полного колебания будет меньше.


\subsubsection{Вопрос 2}

\textbf{$g$ можно определить, измерив время свободного падения и измерив период колебаний маятника. Какой метод даст результат точнее, если пользоваться одним секундомером в обоих случаях?}
\\

		Изучим относительную погрешность определения ${g}$ методом измерения времени падения груза. Ускорение рассчитывается по формуле 

		\begin{equation}
			g=\frac{2H}{t^2}
		\end{equation}

		Относительная погрешность измерения будет равна

		\begin{equation}
			\varepsilon\,(g)=\frac{\Delta\,h}{H}+\frac{2\Delta\,t}{t^2}
		\end{equation}		

		Выразим из известного значения $g=981\frac{\text{см}}{\text{с}^2}$ высоту как функцию времени:

		\begin{equation}
			t^2=\frac{2H}{g}
		\end{equation}		

		Тогда можно посчитать, при какой высоте относительная погрешность не превышает 1.2\%

		\begin{eqnarray}
		\label{e:g}
			\varepsilon\,(g)\le\frac{\Delta\,h}{H}+\frac{2\Delta\,t\cdot{}g}{2H}\\
			0.01\le\frac{0.1}{H}+\frac{0.4\cdot981}{2H}\\
			H\ge\frac{0.1+0.2\cdot981}{0.01}\\
			H\ge19630\text{ см}\approx196\text{ м}
		\end{eqnarray}	

		Из (\ref{e:g}) естественно следует, что погрешность обратно пропорциональна пути свободного падения груза. При $h\le196\text{ м}$ погрешность будет превышать $1.2\%$, таким образом, при сопоставимых длине нити математического маятника (в лабораторных условиях меньше 2 метров) и пути свободного падения груза (при одинаковых погрешностях измерения высоты и времени) точность метода математического маятника будет намного выше.



\subsubsection{Вопрос 3}

\textbf{В каких точках земной поверхности $g$ максимально, в каких минимально и почему?}
\\

		% На Земле ускорение свободного падения складывается из 
		Земля --- эллипсоид вращения, т. е. радиус Земли на полюсе меньше радиуса Земли на экваторе. Ускорение свободного падения рассчитывается по формуле (\ref{g-grav})

		\begin{equation}
		\label{g-grav}
			g=G\cdot\frac{M}{(R+h)^2}, 
		\end{equation}

		где $M$ -- масса земли, $R$ -- радиус земли, $h$ -- высота над поверхностью земли.

		Отсюда ускорение свободного падения на полюсе больше, чем на экваторе ($g = \text{983 см/с}^2$ на полюсе и $g = \text{978 см/с}^2$ на экваторе).

		Кроме того, (\ref{g-grav}) верна при условии изоморфности Земли. В реальности существуют гравитационные аномалии, связанные с неоднородностью её строения, что может быть использовано для поиска полезных ископаемых (гравиразведка).


\subsubsection{Вопрос 4}

\textbf{Чему равно $g$ в центре Земли?}
\\

		В центре земли формула (\ref{g-grav}) неприменима. Представим земной шар как сумму тонкостенных сфер (всего $i$ сфер): так как гравитационное поле аддитивно, то ускорение свободного падения в центре тонкостенной сферы будет интегралом по объему сферы. 

		Попробуем эмпирически представить значение такого интеграла. Разобьем тонкостенную сферу радиуса $R_i$ на маленькие кусочки объема $\Delta{}V$ (следовательно, массы $\Delta{}M$). Тогда относительно центра сферы для каждого такого кусочка найдётся противоположный, лежащий на одном диаметре с данным: то есть для такого диполя можно посчитать $\Delta{}g$

		\begin{equation}
			\Delta{}g=+G\cdot\frac{\Delta{}M}{(R_i)^2}-G\cdot\frac{\Delta{}M}{(R_i)^2}=0
		\end{equation}

		Вся сфера есть сумма всех таких диполей, то $g_i=\sum\Delta\,g=0$. Из аддитивности гравитационного поля следует, что суммарное $g$ есть сумма $g_i$  каждой сферы:

		\begin{equation}
			g=\sum\limits_{1}^{i}g_i=0
		\end{equation}

		Нашли ${g}$ в центре Земли равным нулю.

\subsubsection{Вопрос 5}

\textbf{На какую высоту над Землей нужно подняться, чтобы с помощью приборов, которыми вы пользовались, можно было заметить изменение $g$?}
\\

		При подъеме над поверхностью Земли можно применить формулу (\ref{g-grav}). Для того, чтобы заметить с помощью лабораторного математического маятника изменение $g$, необходимо, чтобы это изменение было больше погрешности измерения (которое нашли равным $1.2\%$)
		\begin{eqnarray*}
			g_0-g_h\ge\varepsilon\,(g)g_0\\
			1-\frac{g_h}{g_0}\ge0.012\\
			1-\frac{G\cdot\frac{M}{(R+H)^2}}{G\cdot\frac{M}{R^2}}\ge0.012\\
			1-\frac{R^2}{(R+H)^2}\ge0.012
			% \frac{g_{h=x}}{g_{h=0}}\ge\frac{G\cdot\frac{M}{(R+x)^2}}{G\cdot\frac{M}{R^2}}=
			% \frac{R^2}{(R+x)^2}
		\end{eqnarray*}
		Положительным решением данного уравнения (вспомним, что требуется найти высоту \textbf{над} Землёй) будет
		\begin{eqnarray*}
			H=38.5\text{ км }
		\end{eqnarray*}

		На высоте $H=38.5$ км замеренное с помощью лабораторного математического маятника ускорение свободного падения будет меньше, чем значение ${g} \pm 1.2\%$ на Земле, следовательно, можно будет сделать вывод о уменьшении ускорения свободного падения (падении напряжённости гравитационного поля).



\end{document}
